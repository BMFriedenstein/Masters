%Chapter 4
\chapter{Results and validation}
\thispagestyle{empty}
\vspace{38em}
\hrulefill
\\
\enquote*{\textit{Quote.}} - Somebody\\
\newpage
\section{Introduction}
\section{Case study: Mine A \color{blue}(Kusasalethu)}
	\subsection{Preamble}
	
	\subsection{Verification of baseline simulation}
	\begin{figure}[h]
		\centering
		\fbox{\input{Graphs/4/KusVerifyFlow/KusVerifyFlow}}
		\caption{Verifying flow.}
		\label{fig: Verification flow kusasalethu}
	\end{figure}

	\begin{figure}[h]
		\centering
		\fbox{\input{Graphs/4/KusVerPressure/KusVerPressure}}
		\caption{Verifying Pressure}
		\label{fig: Verification Pressure kusasalethu}
	\end{figure}

	\begin{figure}[h]
		\centering
		\fbox{\input{Graphs/4/KusVerPower/KusVerPower}}
		\caption{Verifying Power}
		\label{fig: Verification Power kusasalethu}
	\end{figure}

	\subsection{Scenario 1. Refuge bay simulation}
	Tested scenario where all excessive leaking valves are removed.
	Refuge bays savings 1MW E.E.
	
	\begin{figure}[h]
		\centering
		\fbox{\includegraphics[trim =-30cm 0 -20cm 0cm, width=\textwidth]{Images/A/RefBaySim}}
		\caption{Simulation layout for the refuge bay scenario.}
		\label{fig: Refuge bay layout}
	\end{figure}		
	
	
	\subsubsection{Validation of results}
	\subsection{Scenario 2. Closing off levels/stopes}
	\subsection{Summary}
\section{Case study: Mine B \color{blue}(Beatrix 123)}
	\subsection{Preamble}
	\subsection{Scenario 1. Compressor set points}
	\subsection{Scenario 2. Control valves set points}
	\subsection{Summary}
\section{Periodic simulation analysis}
	\subsection{Preamble}
	 Updating the inputs of a simulation periodically could be used to identify significant operational changes or when if the model parameters have drifted to a point where the model requires recalibration. This section will detail the results of an analysis into periodic simulation.
	 \subsection{Method of analysis}
	 
	 \subsection{Results}
	\subsection{Summary}
\section{Potential benefit for SA mines}
\section{Conclusion}