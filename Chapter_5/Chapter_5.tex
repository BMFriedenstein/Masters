%Chapter 5
\chapter{Conclusion}
%\pagenumbering{gobble}
\vspace{38em}

\hrulefill
\\
%\enquote*{\textit{Quote.}} - Somebody\\
\newpage
		\section{Preamble}
		Chapter 5 serves as a conclusion for this dissertation. An overview of the complete dissertation will be provided. This overview will summarise the work done in the proceeding chapters. The limitations of the study will then be discussed. This will lead to discussion and recommendations for future work that can be done in the field.
	 \section{Dissertation overview}
	 The South African mining sector is currently facing significant challenges that pose a risk to profitability of the industry. A pivotal challenge that faces the industry is that of rising operational costs. Energy costs pertain a significant portion of the cost increases as energy tariff increases have constantly surpassed inflation over the passed 10 years.
	 \par
	 Compressed air systems consume the largest portion of energy used in a mine. Compressed air has also been shown to be largely inefficient. It is there for reasoned that the largest energy impact can be achieved through compressed air interventions.
	 \par 
	 Energy intervention in mining compressed air have been performed in the passed. However, compressed air simulation has not been used to its full potential. Using new computer modelling and simulation tools for compressed air systems, the detailed effects of interventions can be identified with minimal risk. This will lead to further energy and cost reductions as well as other potential improvements for a mines operation.
	 \par 
	 An review of background and literature was performed. The purpose of the review was to provide background on mining compressed air networks and to evaluate literature in pertaining to compressed air energy interventions, simulation usage the mining industry and specifically simulation usage in compressed air systems.
	 \par 
	 Using the findings in from the literature review, a methodology was developed for compressed air simulation. The methodology describes the a simulation procedure in three steps summarised as: investigate the system, develop a simulation model and execute the simulation scenarios.
	 \par 
	 The simulation methodology was then validated through implementationon case studies. The studies were implemented on two separate mine compressed air systems. A full system investigation was performed in each study. From the data and understand obtained for the investigation, simulation models were developed for both systems.
	 \par
	 \emph{...Incomplete....}
	 \section{Limits of this study}
	 \section{Recommendations for future studies}
	 
	 