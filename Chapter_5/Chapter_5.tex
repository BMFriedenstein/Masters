%Chapter 5
\chapter{Conclusion}
\thispagestyle{empty}
\vspace{40em}
\hrulefill
\\
%\enquote*{\textit{Quote.}} - Somebody\\
\newpage
		\section{Preamble}
		Chapter 5 serves as a conclusion for this dissertation. An overview of the complete dissertation will be provided. This overview will summarise the work done in the proceeding chapters. The limitations of the study will then be discussed. This overview will lead to a discussion of recommendations for future work that can be done in the field.
	 \section{Dissertation overview}
		The South African mining sector is currently facing significant challenges that pose a risk to the profitability of the industry. A central challenge that faces the industry is that of rising operational costs. Energy costs contribute a significant portion of the cost increases as energy tariff increases have constantly surpassed inflation over the past ten years.
	 \par
		Compressed air systems consume the largest portion of energy used in a mine. Compressed air has also been shown to be largely inefficient. It is there for reasoned that the greatest energy impact can be achieved through compressed air interventions.
	 \par 
		Energy interventions in mining compressed air have been performed in the past. However, compressed air simulation has not been used to its full potential.  The specific effects of interventions can be identified with minimal risk using new computer modelling and simulation tools for compressed air systems; This will lead to further energy and cost reductions as well as other potential improvements for the operation of a mine.
	 \par 
	 A review of background and literature was performed. The purpose of the review was to provide background on mining compressed air networks and to evaluate literature in pertaining to compressed air energy interventions, simulation usage the mining industry and specifically simulation usage in compressed air systems.
	 \par 
		A methodology was developed for compressed air simulation using the findings in from the literature review. The method describes the simulation procedure in three steps summarised as follows: investigate the system, develop a simulation model and execute the simulation scenarios.
	 \par 
The simulation methodology implemented with case studies. The studies were performed on two different mines compressed air systems. A full system investigation was conducted in each study. From the data and information obtained from the investigation, simulation models were developed for both systems. The models were verified using measurement data from the physical system. The models were used to simulate various operation scenarios.
	 \par 
In Case study 1, two scenarios were simulated. The results first scenario, reducing compressor setpoints, showed potential power reduction of 0.46 MW P.C. which would result in a yearly energy cost saving of R0.37M. Scenario 2, reducing underground control valve pressure, showed a potential power reduction of 1.0 MW P.C. which would result in a yearly energy cost saving of R0.91M. The result of the simulation was validated by comparison with the actual results from the test results from the physical system.
	 \par
	 In Case study 2, two scenarios were simulated. The first scenario looked at reducing refuge bay leaks. The results showed a potential energy efficiency improvement of 0.92 MW, which would result in an annual energy cost saving of R5.17M. A further pressure improvement of 15 kPa during the drilling time was identified. In Scenario 2, optimising peak time demand through station and stope control was investigated. The results showed a potential power reduction of up to 2.0 MW \gls{pc} could be obtained. The optimisation would lead to energy cost saving of R0.91M \gls{pa}.
	 \par 
	 In case study 3, an analysis into repeated, periodic simulations was done. The investigation aimed to check the validity of simulation models and to identify when major shifts in a compressed air system’s operation occur. A faulty power meter was identified. However, from the pressure and flow process parameters, the analysis showed that the model accurately represented the system over the period of repeated simulations.
	 \par
	 Implementation of the simulation methodology developed in this study to other gold and PGM mine compressed air systems would significant energy cost and operation improvement for the industry. It was estimated that up to a 30 MW energy efficiency improvements could be achieved. This increase in efficiency would lead to cost savings up to R240M p.a for the industry. 
	 %\section{Limits of this study}
	 \section{Recommendations for future studies}
	 \begin{itemize}
	 	\item Simulation methodology for other mining systems, ventilation, water reticulation 
	 	\item A sensitivity analysis to analyse the accuracy of simulations with different boundary choices.
	 	\item Further analysis of error calculations and verification methods.
	 	\item Quantify financial benefits of pressure improvements for drilling
	 	\item more...
	 	
	 \end{itemize}
	 \textit{Unfinished}
	 