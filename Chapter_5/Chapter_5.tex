%Chapter 5
\chapter{Conclusion}
\thispagestyle{empty}
\vspace{40em}
\hrulefill
\\
%\enquote*{\textit{Quote.}} - Somebody\\
\newpage
		\section{Preamble}
		Chapter 5 serves as the conclusion of this dissertation and it provides an overview of the complete dissertation. This overview summarises the work done in the preceding chapters and discusses the limitations of the study as well as recommendations for future work to be done in the field.
	 \section{Dissertation overview}
		The South African mining sector is currently facing significant challenges that pose a risk to the profitability of the industry. A central challenge that faces the industry is that of rising operational costs. Energy costs constitute a significant portion of the cost increases as energy tariff increases consistently exceeded inflation over the past ten years.
	 \par
		Compressed air systems do not only consume the largest portion of energy used in a mine, but they were also shown to be largely inefficient. It is therefore reasoned that the greatest energy impact can be achieved through interventions in respect of compressed air networks.
	 \par 
		Energy interventions in mining compressed air systems have been performed in the past. However, compressed air simulation has not been used to its full potential. The specific effects of interventions imply minimal risk, by using new computer modelling and simulation tools for compressed air systems. This will lead to further energy and cost reductions as well as other potential improvements in the operation of a mine.
	 \par 
	 A review of background and literature was performed. The purpose of the review was to provide background on mining compressed air networks and to evaluate literature pertaining to compressed air energy interventions, as well as to simulation usage in the mining industry in general and compressed air systems in particular.
	 \par 
		A methodology was developed for compressed air simulation, using the findings that emerged from the literature review. The method describes the simulation procedure in three steps, namely: investigate the system; develop a simulation model; and execute the simulation scenarios.
	 \par 
The simulation methodology was implemented with case studies that were performed on the compressed air systems of two different mines. A full system investigation was conducted in each study. From the data and information obtained from the investigation, simulation models were developed for both systems. The models were verified using measurement data from the physical system and the were used to simulate various operational scenarios.
	 \par 
In Case Study 1, two scenarios were simulated. The results of the first scenario --- reducing compressor  set-points --- showed a potential power reduction of 0.46 $MW$ P.C., which would result in a yearly energy cost saving of R0.37m. Scenario 2 --- reducing underground control valve pressure --- showed a potential power reduction of 1.0 $MW$ P.C., which would result in a yearly energy cost saving of R0.91m. The result of the simulation was validated by comparing it with the actual results of the test conducted on the physical system.
	 \par
	 In Case Study 2, two scenarios were simulated. The first scenario looked at reducing refuge bay leaks. The results showed a potential energy efficiency improvement of 0.92 $MW$, which would result in an annual energy cost saving of R5.17m. A further pressure improvement of 15 $kPa$ during the drilling time was identified. In Scenario 2, optimising peak-time demand through station and in-stope control was investigated. The results showed that a potential power reduction of up to 2.0 $MW$ \gls{pc} could be obtained. This optimisation would lead to energy cost saving of R0.91m \gls{pa}.
	 \par 
	 In Case Study 3, repeated, periodic simulations were analysed. The investigation aimed to check the validity of simulation models and to identify when major shifts in a compressed air system’s operation occur. A faulty power meter was identified. However, based the pressure and flow process parameters, the analysis showed that the model accurately represented the system over the period of repeated simulations.
	 \par
	 Implementing the simulation methodology developed in this study in other gold and PGM mines' compressed air systems would  bring about significant energy cost savings and operational improvements for the industry. It was estimated that up to a 30 $MW$ energy efficiency improvement could be achieved. This increase in efficiency would lead to cost savings of up to R240m p.a for the industry. 
	 %\section{Limits of this study}
	 \section{Recommendations for future studies}
	 This study focused on operational improvements in compressed air systems as they offered the largest scope for energy improvement. However, an integrated simulation approach could also lead to significant operational improvements in other large industrial systems, such as ventilation and water reticulation.
	 \par
	 \clearpage
	 Simulation model details well as varying air properties is dependent on system boundary selection. Since the current study did not investigate the actual effect of different boundary selections with regard to simulation accuracy, there is scope for a sensitivity analysis to analyse the accuracy of simulations with different boundary choices.
	 \par
	 A major limiting factor in integrated simulations of large systems is the availability of reliable data. A study that focuses on the value of instrumentation for simulation and system improvement identification is recommended. The study should investigate the availability of data in mining systems and the effect and the  relation with operational efficiency and improvement measures.
	 \par 
	 In Case Study 2, an additional pressure benefit was identified in one of the simulated scenarios. It is recommended that future studies investigate the financial impact of pressure improvements that may result from quicker/more efficient drilling. Finally, a future study should be performed to validate the simulation results achieved in Case Study 2. This can be achieved through practical tests executed on the physical system.
