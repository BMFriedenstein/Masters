%Chapter 1
\chapter{Introduction and background}  %J.H Marais, C.J.R. Kriel
\pagenumbering{arabic}
\thispagestyle{empty}
\vspace{40em}
\hrulefill
\\
\enquote{\textit{Precisely one of the most gratifying results of intellectual evolution is the continuous opening up of new and greater prospects.}} -  Nikola Tesla\\
\newpage

\section{Preamble}
Chapter 1 starts off by discussing the background of deep-level mining in South Africa. Next, it examines the need to reduce operational costs. It focuses on reducing the energy consumption of compressed air systems and it attends to compressed air operation and energy interventions. Simulations and their value in the industry are reviewed, and this leads to a statement of te problem and the objectives of the study. The chapter concludes with an overview of the dissertation.
\section{Background on deep level mining}
	\subsection{Mining profitability}
	
	 	Various technical, economic, social and operational challenges are posing a risk to the profitability of the South African mining sector. One of the challenges faced by the sector is a rise in the cost of operations \cite{neingo2016trends}.
	 	\par
	 	\footnotetext[1]{Eskom, \enquote{Revenue application - multi-year price determination 2013/14 to 2017/18 (mypd3),} [Online] \url{http://www.eskom.co.za/CustomerCare/MYPD3/Documents/NersaReasonsforDecision.pdf, 2013}, [Accessed 22 March 2017].}
	 	
		A factor that has in recent years contributed considerably to the increase in operational costs in South African gold mines is the rising of electricity costs. As shown in \Cref{fig: Eskom tariffs}, the general cost of electricity has increased at a rate higher than inflation since 2008\footnotemark[1].
		
		\footnotetext[2]{inflation.eu, \enquote{Historic inflation South Africa,} [Online] \url{http://www.inflation.eu/inflation-rates/south-africa/historic-inflation/cpi-inflation-south-africa.aspx}, [Accessed 25 March 2017].}
		\begin{figure}[h]
			\centering
			\fbox{% GNUPLOT: LaTeX picture with Postscript
\begingroup
  \makeatletter
  \providecommand\color[2][]{%
    \GenericError{(gnuplot) \space\space\space\@spaces}{%
      Package color not loaded in conjunction with
      terminal option `colourtext'%
    }{See the gnuplot documentation for explanation.%
    }{Either use 'blacktext' in gnuplot or load the package
      color.sty in LaTeX.}%
    \renewcommand\color[2][]{}%
  }%
  \providecommand\includegraphics[2][]{%
    \GenericError{(gnuplot) \space\space\space\@spaces}{%
      Package graphicx or graphics not loaded%
    }{See the gnuplot documentation for explanation.%
    }{The gnuplot epslatex terminal needs graphicx.sty or graphics.sty.}%
    \renewcommand\includegraphics[2][]{}%
  }%
  \providecommand\rotatebox[2]{#2}%
  \@ifundefined{ifGPcolor}{%
    \newif\ifGPcolor
    \GPcolortrue
  }{}%
  \@ifundefined{ifGPblacktext}{%
    \newif\ifGPblacktext
    \GPblacktextfalse
  }{}%
  % define a \g@addto@macro without @ in the name:
  \let\gplgaddtomacro\g@addto@macro
  % define empty templates for all commands taking text:
  \gdef\gplbacktext{}%
  \gdef\gplfronttext{}%
  \makeatother
  \ifGPblacktext
    % no textcolor at all
    \def\colorrgb#1{}%
    \def\colorgray#1{}%
  \else
    % gray or color?
    \ifGPcolor
      \def\colorrgb#1{\color[rgb]{#1}}%
      \def\colorgray#1{\color[gray]{#1}}%
      \expandafter\def\csname LTw\endcsname{\color{white}}%
      \expandafter\def\csname LTb\endcsname{\color{black}}%
      \expandafter\def\csname LTa\endcsname{\color{black}}%
      \expandafter\def\csname LT0\endcsname{\color[rgb]{1,0,0}}%
      \expandafter\def\csname LT1\endcsname{\color[rgb]{0,1,0}}%
      \expandafter\def\csname LT2\endcsname{\color[rgb]{0,0,1}}%
      \expandafter\def\csname LT3\endcsname{\color[rgb]{1,0,1}}%
      \expandafter\def\csname LT4\endcsname{\color[rgb]{0,1,1}}%
      \expandafter\def\csname LT5\endcsname{\color[rgb]{1,1,0}}%
      \expandafter\def\csname LT6\endcsname{\color[rgb]{0,0,0}}%
      \expandafter\def\csname LT7\endcsname{\color[rgb]{1,0.3,0}}%
      \expandafter\def\csname LT8\endcsname{\color[rgb]{0.5,0.5,0.5}}%
    \else
      % gray
      \def\colorrgb#1{\color{black}}%
      \def\colorgray#1{\color[gray]{#1}}%
      \expandafter\def\csname LTw\endcsname{\color{white}}%
      \expandafter\def\csname LTb\endcsname{\color{black}}%
      \expandafter\def\csname LTa\endcsname{\color{black}}%
      \expandafter\def\csname LT0\endcsname{\color{black}}%
      \expandafter\def\csname LT1\endcsname{\color{black}}%
      \expandafter\def\csname LT2\endcsname{\color{black}}%
      \expandafter\def\csname LT3\endcsname{\color{black}}%
      \expandafter\def\csname LT4\endcsname{\color{black}}%
      \expandafter\def\csname LT5\endcsname{\color{black}}%
      \expandafter\def\csname LT6\endcsname{\color{black}}%
      \expandafter\def\csname LT7\endcsname{\color{black}}%
      \expandafter\def\csname LT8\endcsname{\color{black}}%
    \fi
  \fi
    \setlength{\unitlength}{0.0500bp}%
    \ifx\gptboxheight\undefined%
      \newlength{\gptboxheight}%
      \newlength{\gptboxwidth}%
      \newsavebox{\gptboxtext}%
    \fi%
    \setlength{\fboxrule}{0.5pt}%
    \setlength{\fboxsep}{1pt}%
\begin{picture}(9360.00,4032.00)%
    \gplgaddtomacro\gplbacktext{%
      \colorrgb{0.00,0.00,0.00}%
      \put(682,1144){\makebox(0,0)[r]{\strut{}$0$}}%
      \colorrgb{0.00,0.00,0.00}%
      \put(682,1422){\makebox(0,0)[r]{\strut{}$5$}}%
      \colorrgb{0.00,0.00,0.00}%
      \put(682,1701){\makebox(0,0)[r]{\strut{}$10$}}%
      \colorrgb{0.00,0.00,0.00}%
      \put(682,1979){\makebox(0,0)[r]{\strut{}$15$}}%
      \colorrgb{0.00,0.00,0.00}%
      \put(682,2258){\makebox(0,0)[r]{\strut{}$20$}}%
      \colorrgb{0.00,0.00,0.00}%
      \put(682,2536){\makebox(0,0)[r]{\strut{}$25$}}%
      \colorrgb{0.00,0.00,0.00}%
      \put(682,2814){\makebox(0,0)[r]{\strut{}$30$}}%
      \colorrgb{0.00,0.00,0.00}%
      \put(682,3093){\makebox(0,0)[r]{\strut{}$35$}}%
      \colorrgb{0.00,0.00,0.00}%
      \put(682,3371){\makebox(0,0)[r]{\strut{}$40$}}%
      \colorrgb{0.00,0.00,0.00}%
      \put(814,924){\makebox(0,0){\strut{}$2006$}}%
      \colorrgb{0.00,0.00,0.00}%
      \put(2172,924){\makebox(0,0){\strut{}$2008$}}%
      \colorrgb{0.00,0.00,0.00}%
      \put(3530,924){\makebox(0,0){\strut{}$2010$}}%
      \colorrgb{0.00,0.00,0.00}%
      \put(4888,924){\makebox(0,0){\strut{}$2012$}}%
      \colorrgb{0.00,0.00,0.00}%
      \put(6246,924){\makebox(0,0){\strut{}$2014$}}%
      \colorrgb{0.00,0.00,0.00}%
      \put(7604,924){\makebox(0,0){\strut{}$2016$}}%
      \colorrgb{0.00,0.00,0.00}%
      \put(8962,924){\makebox(0,0){\strut{}$2018$}}%
    }%
    \gplgaddtomacro\gplfronttext{%
      \csname LTb\endcsname%
      \put(176,2257){\rotatebox{-270}{\makebox(0,0){\strut{}\% increase}}}%
      \put(4888,594){\makebox(0,0){\strut{}Year}}%
      \put(4888,3701){\makebox(0,0){\strut{}Electricity price increases in South Africa}}%
      \csname LTb\endcsname%
      \put(6770,393){\makebox(0,0)[r]{\strut{}Eskom general tariff increases (\%)}}%
      \csname LTb\endcsname%
      \put(1493,1562){\makebox(0,0){\strut{}6.0}}%
      \put(2172,2759){\makebox(0,0){\strut{}27.5}}%
      \put(2851,2970){\makebox(0,0){\strut{}31.3}}%
      \put(3530,2608){\makebox(0,0){\strut{}24.8}}%
      \put(4209,2664){\makebox(0,0){\strut{}25.8}}%
      \put(4888,2118){\makebox(0,0){\strut{}16.0}}%
      \put(5567,1673){\makebox(0,0){\strut{}8.0}}%
      \put(6246,1673){\makebox(0,0){\strut{}8.0}}%
      \put(6925,1673){\makebox(0,0){\strut{}8.0}}%
      \put(7604,1673){\makebox(0,0){\strut{}8.0}}%
      \put(8283,1673){\makebox(0,0){\strut{}8.0}}%
      \csname LTb\endcsname%
      \put(6770,173){\makebox(0,0)[r]{\strut{}Inflation rate in south Africa (\%)}}%
    }%
    \gplbacktext
    \put(0,0){\includegraphics{Graphs/1/Eskom/Eskom}}%
    \gplfronttext
  \end{picture}%
\endgroup
}
			\caption[Electricity price increases compared to the inflation rate in South Africa]{Electricity price increases\protect\footnotemark[1] compared to the inflation rate in South Africa\protect\footnotemark[2] between 2007 and 2017}
			\label{fig: Eskom tariffs}
		\end{figure}
	
		\par
		In addition to rising electricity costs, gold ore grades of South African mines have fallen substantially in recent decades \cite{mudd2007global}. As ore grades decline, the energy utilised per unit of metal increases exponentially \cite{muller2010numerical}. 
		\par
		Furthermore, mines need to extend their operations deeper into the earth to reach the valuable ore reefs\footnote{Wall Street Journal, \enquote{SA Miners Dig Deeper to Extend Gold Veins' Life Spans,} [Online] \url{https://www.wsj.com/articles/SB10001424052748703584804576144062424424614}, [Accessed 25 March 2016].}. Therefore, mines require significantly more energy per unit of metal produced. This combination of tariff increases and increased energy usage per unit has led to a significant rise in mining operation costs.
	\subsection{Process of a deep-level mine}
	South Africa's mines are some of the deepest in the world. Some mine shafts are reaching depths greater than 4000 $m$ below the surface \cite{vosloo2012case}. The process of extracting ore at this depth is dependent on a number of essential services, mainly cooling and ventilation, pumping, compressed air and hoisting (see \Cref{fig: Mining Layout}).
	\par 
	\begin{figure}[h!]
		\centering
		\fbox{\includegraphics[width=0.98\textwidth]{Graphs/1/Layout/Layout.pdf}}
		\caption{Schematic representation of the mining processes}
		\label{fig: Mining Layout}
	\end{figure}
	
	 Cooling and ventilation systems are required to maintain a safe working temperature underground. Pumping is critical to remove service and fissure water, as well as to prevent flooding. Compressed air is used to power underground drills and machines. Compressed air is used as it is safer than other available energy sources for in-stope drilling. As these drills operate using air, there is no added risk for fire.
	 \par
	  A hoisting system is needed to bring the ore to the surface and to transport workers inside the mine. The hoisted materials are transferred to a plant for processing.  
		
		\subsection{Energy usage of mining services}
		\footnotetext[1]{Eskom, \enquote{The energy efficiency series - towards an energy efficient mining sector,} [Online] \url{http://www.eskom.co.za/sites/idm/Documents/121040ESKD}, February 2010, [Accessed 19 March 2017].}
		
			In South Africa, the mining industry utilises approximately 15\% of the national electricity supplier's annual output, of which gold and platinum mines use  close to 80\%\footnotemark[1]. \Cref{fig: Energy Split} illustrates the division of energy use within the mining industry. The chart indicates that compressed air systems utilise the biggest amount of energy within a mine. As compressed air is the highest energy user and due to the low efficiency of compressed air systems, it is deduced that energy use can be reduced most effectively through the implementation of energy-saving interventions in respect of compressed air systems.
			 %%%%%%%%%%% Explain other parameters on graph
			\begin{figure}[h]
				\centering
				\fbox{\input{Graphs/1/distribution/distribution}}
				\caption[The percentage of energy consumption for each type of mining system]{The percentage of energy consumption for each type of mining system \cite{le2005energy}}
				\label{fig: Energy Split}
			\end{figure}
\section{Compressed air systems in mining}
	\subsection{Compressed air in operation}\label{key}
		Primarily due to their reliability, versatilely and ease of use, the South African mining industry has installed extensive compressed air networks. These systems can have compressors with capacities of up to 15 megawatt ($MW$)\cite{Marais2012PhD}. However, the supply of compressed air is a high-energy demanding and costly process \cite{padachi2009energy}.
		\par 
		The energy used for compressed air production amounts to between 9\% and 20\% of the total mining energy consumption\footnotemark[1] \cite{du2011development}.  Unfortunately, the compression process is highly inefficient and \cite{fraser2008saving} and \cite{yang2009air} have shown that the efficiency of the process of converting electrical energy to power pneumatic drills is as low as 4\% and between 5\% and 20\%  respectively. The efficiency of these machines is quite low when compared to alternative rock drills such as electric, oil, electro-hydraulic and hydro-powered drills that convert energy at an efficiency of between 20-31\% \cite{fraser2008saving}, \cite{vanTonder2010Masters}. 
		\par
		%%%%%%%%%%%%%%%%%%%%%%%%%%%%%%%%%%%%%%
		\clearpage
		%%%%%%%%%%%%%%%%%%%%%%%%%%%%%%%%%%%%%%
		Sizeable compressed air systems are largely inefficient. Internationally, the expected energy-savings potential of large compressed air networks has been proven to be 15\% \cite{neale2009compressed}. Marais \cite{marais2013simplification} calculates that energy efficiency improvements could lead to energy and cost savings of between 30\% and 40\%. 
		
	\subsection{Inefficiencies found in compressed air systems}
		Compressed air distribution networks in the mining industry consist of multiple compressors and working areas are up to eight kilometres away from the source on the surface \cite{Marais2012PhD}. Due to their size and complexity, these systems are prone to significant energy losses \cite{Marais2012PhD}.
		\par 
		Compressed air leakage accounts for as much as 35\% of the energy losses of a compressed air network \cite{Lawrence2004Improving}. Other systemic losses include faulty valves, pipe diameter fluctuations, obstructed air compressor intake filters and inefficient compressors. 	
		\par
		Leakage and inefficiency detection strategies are not commonly pursued in the South African mining industry \cite{vanTonder2010Masters}. However, many mines do perform leak inspections either internally or by an outside company. In these inspections, an ultrasonic detector is often used to locate the leak. Alternatively, some mines employ the \enquote{walk and listen} method to identify leaks from the audible sound that they produce \cite{vanTonder2010Masters}. Once the inspection has been completed, the findings, including the locations and estimated cost of all identified leaks, are reported.
		
	\subsection{Compressed air savings interventions}
		In Chapter 2, successful compressed air intervention on minning systems are discussed on the basis of a literature review. The available literature reports that energy-saving interventions on compressed air systems implemented one or a combination of following strategies \cite{Snyman2011Masters}:
		\begin{itemize}
			\item Reducing leaks
			\item Reducing demand
			\item Reducing unauthorised air usage
			\item Increasing supply efficiency
			\item Optimising supply
		\end{itemize}
	 A combination of energy strategies often leads to the most savings \cite{Marais2012PhD}. 
	 \par 
	 Once an energy-saving measure has been identified, it is necessary to make estimations to determine the potential costs and benefits of the intervention. The estimations are typically performed using first principle calculations, simplified mathematical models and practical tests where possible. 
	 \par 
	 However, new tools have enabled the quick and accurate development of a compressed air model. Through simulations, accurate estimations can be obtained quickly, with no risk and at comparatively low resource requirement.
\section{Use of simulation in industry }
	\subsection{Background on industrial simulation}
	
		Continuous improvement in computing hardware have led to major advancement in software technology. Consequently, computational simulation has become an increasingly valuable tool to be used in many industries \cite{kocsis2003integration}.
		\par 
		In \textit{ Handbook of Simulation: Principles, methodology, advances, applications, and practice,} \cite{banks1998handbook} the advantages of the use of simulation in the industry are listed: %%% Chapter 1 of book
		\begin{itemize}
			\item The ability to test new policies, operating procedures and methods without disrupting the actual system.
			\item The means to identify problems in complex systems by gathering insight about the interactions within the system.
			\item The facility to compress or expand time to investigate phenomena thoroughly.
			\item The capability to determine the limits and constraints within a system.
			\item The potential to build consensus about proposed designs or modifications.
		\end{itemize}

	\subsection{Simulation usage in compressed air optimisation}
		Simulation is used to test and identify energy and operational improvement modifications in mining compressed air systems. However, in the past  the production of complex models for mining systems was not feasible as the simulation software was too cumbersome fo use for large compressed air systems use and required often unattainable obtain data inputs \cite{marais2013simplification}. 
		\par 
		\subsubsection{Simplified \enquote{vessel} model}
		Before new software tools allowed for the development of detailed mining compressed air simulation models, Marais \cite{Marais2012PhD}, \cite{marais2013simplification} created a simplified compressed air model to estimate and quantify the performance of potential energy interventions. Marais simplified the mining compressed air system, and compared the network to an air source and a vessel with many leaks. The simplified model is illustrated in \Cref{fig:Marais vessel model}.
		\begin{figure}[h!]
			\centering
			\fbox{\includegraphics[trim= 0 0.4cm 0 0.8cm, width=0.98\textwidth]{Images/2/marais/Vessel}}
			\caption[Simplified compressed air netowrk model]{Simplified compressed air network model. (adapted from Marais \cite{Marais2012PhD})}
			\label{fig:Marais vessel model}
		\end{figure}
		\par 
		A calculation methodology was developed to quickly estimate the expected energy-saving impact on the system.Based on this methodology, energy-saving estimations and rules were designed as listed in \Cref{table: Rules of thumb}. \\
		\par 
		\begin{table}[h]
			\caption[Summary of energy saving estimation rules]{Summary of energy saving estimation rules \cite{Marais2012PhD}}
			\centering
			\begin{tabular}{p{0.4\textwidth}p{0.01\textwidth}p{0.5\textwidth}}
				\hline
				Intervention && Estimation rule\\
				\hhline{===} 
				Reducing compressor deliver pressure & & $x \%$ pressure reduction $\propto$ (1.6 to 1.8)$\cdot x\%$ power reduction \newline \\
				Reduce control valve pressure & &$x \%$ pressure reduction $\propto$ $p\cdot x\%$ power reduction. \newline \newline Where p is the valves' relative flow contribution to the system \newline \\
				Reduction of flow && $x \%$ flow reduction $\propto x \%$ power reduction \newline\\
				\hline
			\end{tabular} 
			\label{table: Rules of thumb}
		\end{table}
		There is not a high degree of precision in this approach as specific details regarding the air network are not taken into account. The simplified approach cannot be used to estimate more complex scenarios. The method also does not estimate other potential benefits of interventions such as pressure delivery improvements.
		
		\subsubsection{Simplified air network model}
		Kriel \cite{Kriel2014Masters} used simulation to estimate the performance of energy projects on mine compressed air systems. The KYPipe GAS software tool was used to develop simulation models for these systems. Kriel simplified the air networks for the simulations to a single compressor that represents the supply processes and an outlet flow to each underground level in the network. The model is shown graphically in \Cref{fig:kriel model}.
		\begin{figure}[h!]
			\centering
			\fbox{\includegraphics[trim = -2cm 0.5cm -2cm 0.5cm ,width=0.98\textwidth]{Images/2/kriel/model}}
			\caption[Simplified system model]{Simplified system model (adapted from Kriel \cite{Kriel2014Masters})}
			\label{fig:kriel model}
		\end{figure}
		\par 
		The simulation was performed to quantify the savings from underground network interventions that had been designed to reduce flow to the network. The estimated savings from the simulations varied between 10 and 25\% compared to the actual performance of the interventions. Like the \enquote{vessel} model discussed previously, the simplified air network model cannot be used to estimate the energy-savings potential of more complex scenarios.
		\par 
		 The simulation procedure in this study could be improved by using a more detailed model and a more precise verification method. The method could lead to more accurate predictions of savings. 
		\par
		Planned manual measurements, estimations and new software technologies can be used to develop more detailed compressed air models \cite{Bredenkamp2015Challeges}, \cite{Mare2017Evaluating}. Using a structured procedure may allow for the development of more detailed and accurate mine compressed air simulations.
		
		%A standard procedure for developing and verifying simulation has also not been developed.

\section{Problem statement and objectives}
	\subsection{Problem statement}
 		Rising expenditure and falling ore grades are compelling the mining industry to reduce operational costs. Research has shwon that the industry can reduce energy costs significant through interventions in compressed air systems. Unfortunately, manual testing of these interventions is risky and cumbersome.
 		\par
 		Computer modelling and simulation of compressed air systems can be used to quantify and prioritise operational interventions with minimal risk. However, integrated simulations have not been used to their full potential in compressed air system studies in the past. With new tools available to develop detailed simulation models, the energy and operational efficiency of mining compressed air systems can be improved, and this may lead to significant energy and cost savings as well as other potential improvements in mining operations.
 		\par 
 		Thus, a need exists for an integrated compressed air simulation approach to identify energy and operational improvements for mines.
 			\subsection{Research objectives}
 			
 			%%%%%% Not Good
		The primary aim of this study was to create energy savings opportunities through the identification of operational improvements in mining compressed air systems. A simulation process was developed to achieve this goal. The other objectives of this study were the following:
		\begin{itemize}
			\item Develop an integrated approach to develop and implement compressed air simulations
			\item Use simulations to apply and rank compressed air operation interventions
			\item Model mining compressed air networks and components accurately
			\item Develop an approach to verify compressed air simulation models
		\end{itemize}
		
\section{Dissertation overview}
	\paragraph{Chapter 1} \hspace{0.4cm} -- This chapter served as an introduction to the dissertation. It provided the relevant background in mining, compressed air and simulation to establish a suitable problem statement. The objectives of the study were subsequently outlined.
	\paragraph{Chapter 2} \hspace{0.4cm} -- Chapter 2 next provides a review of the literature and necessary background relevant to the study. The chapter starts by presenting some background on mining compressed air networks and then conducts a review of compressed air energy interventions. A review of simulation usage in the mining industry comes next, and finally, a review  is given of simulation usage in mining compressed air systems.
	\paragraph{Chapter 3} \hspace{0.4cm} -- Chapter 3 provides a simulation methodology that outlines the processes used to investigate a compressed air system, to develop and calibrate simulation models, and finally to implement simulations and obtain results.
	\paragraph{Chapter 4} \hspace{0.4cm} -- Chapter 4 provides validation of the methodology through case study results, based on the results of three case studies. Finally, the impact of large-scale implementation of the simulation methodology is discussed.
	\paragraph{Chapter 5} \hspace{0.4cm} -- Chapter 5 concludes the dissertation. The chapter also provides recommendations for further studies on compressed air simulation.
	%\texttt{Describe (in approximately one sentence each) the contents of \\each of the dissertation chapters. No results here.}
