%Chapter 1
\chapter{Introduction and background}  %J.H Marais, C.J.R. Kriel
\pagenumbering{gobble}
\vspace{38em}
\hrulefill
\\
\enquote*{\textit{Quote.}} - Somebody\\
\newpage
\pagenumbering{arabic} 
\setcounter{page}{2}

\section{Preamble}
\section{Background on deep level mining}
\subsection{Mining profitability}
 Various technical, economic, social and operational challenges are posing a risk to the profitability of the South African mining sector. One of the challenges the sector faces  is a rise in the cost of operation \cite{neingo2016trends}.\par
%\paragraph*{Falling ore grades}\leavevmode\\
A considerable factor that is contributing to the rise of operational costs in South African gold mines has been the increase in electricity costs. As shown in figure \ref{fig: Eskom tariffs}, the general cost of electricity has increased at a rate greater than inflation since 2008 \cite{Eskom2013Tariffs}.
\begin{figure}[h]
	\centering
	\fbox{% GNUPLOT: LaTeX picture with Postscript
\begingroup
  \makeatletter
  \providecommand\color[2][]{%
    \GenericError{(gnuplot) \space\space\space\@spaces}{%
      Package color not loaded in conjunction with
      terminal option `colourtext'%
    }{See the gnuplot documentation for explanation.%
    }{Either use 'blacktext' in gnuplot or load the package
      color.sty in LaTeX.}%
    \renewcommand\color[2][]{}%
  }%
  \providecommand\includegraphics[2][]{%
    \GenericError{(gnuplot) \space\space\space\@spaces}{%
      Package graphicx or graphics not loaded%
    }{See the gnuplot documentation for explanation.%
    }{The gnuplot epslatex terminal needs graphicx.sty or graphics.sty.}%
    \renewcommand\includegraphics[2][]{}%
  }%
  \providecommand\rotatebox[2]{#2}%
  \@ifundefined{ifGPcolor}{%
    \newif\ifGPcolor
    \GPcolortrue
  }{}%
  \@ifundefined{ifGPblacktext}{%
    \newif\ifGPblacktext
    \GPblacktextfalse
  }{}%
  % define a \g@addto@macro without @ in the name:
  \let\gplgaddtomacro\g@addto@macro
  % define empty templates for all commands taking text:
  \gdef\gplbacktext{}%
  \gdef\gplfronttext{}%
  \makeatother
  \ifGPblacktext
    % no textcolor at all
    \def\colorrgb#1{}%
    \def\colorgray#1{}%
  \else
    % gray or color?
    \ifGPcolor
      \def\colorrgb#1{\color[rgb]{#1}}%
      \def\colorgray#1{\color[gray]{#1}}%
      \expandafter\def\csname LTw\endcsname{\color{white}}%
      \expandafter\def\csname LTb\endcsname{\color{black}}%
      \expandafter\def\csname LTa\endcsname{\color{black}}%
      \expandafter\def\csname LT0\endcsname{\color[rgb]{1,0,0}}%
      \expandafter\def\csname LT1\endcsname{\color[rgb]{0,1,0}}%
      \expandafter\def\csname LT2\endcsname{\color[rgb]{0,0,1}}%
      \expandafter\def\csname LT3\endcsname{\color[rgb]{1,0,1}}%
      \expandafter\def\csname LT4\endcsname{\color[rgb]{0,1,1}}%
      \expandafter\def\csname LT5\endcsname{\color[rgb]{1,1,0}}%
      \expandafter\def\csname LT6\endcsname{\color[rgb]{0,0,0}}%
      \expandafter\def\csname LT7\endcsname{\color[rgb]{1,0.3,0}}%
      \expandafter\def\csname LT8\endcsname{\color[rgb]{0.5,0.5,0.5}}%
    \else
      % gray
      \def\colorrgb#1{\color{black}}%
      \def\colorgray#1{\color[gray]{#1}}%
      \expandafter\def\csname LTw\endcsname{\color{white}}%
      \expandafter\def\csname LTb\endcsname{\color{black}}%
      \expandafter\def\csname LTa\endcsname{\color{black}}%
      \expandafter\def\csname LT0\endcsname{\color{black}}%
      \expandafter\def\csname LT1\endcsname{\color{black}}%
      \expandafter\def\csname LT2\endcsname{\color{black}}%
      \expandafter\def\csname LT3\endcsname{\color{black}}%
      \expandafter\def\csname LT4\endcsname{\color{black}}%
      \expandafter\def\csname LT5\endcsname{\color{black}}%
      \expandafter\def\csname LT6\endcsname{\color{black}}%
      \expandafter\def\csname LT7\endcsname{\color{black}}%
      \expandafter\def\csname LT8\endcsname{\color{black}}%
    \fi
  \fi
    \setlength{\unitlength}{0.0500bp}%
    \ifx\gptboxheight\undefined%
      \newlength{\gptboxheight}%
      \newlength{\gptboxwidth}%
      \newsavebox{\gptboxtext}%
    \fi%
    \setlength{\fboxrule}{0.5pt}%
    \setlength{\fboxsep}{1pt}%
\begin{picture}(9360.00,4032.00)%
    \gplgaddtomacro\gplbacktext{%
      \colorrgb{0.00,0.00,0.00}%
      \put(682,1144){\makebox(0,0)[r]{\strut{}$0$}}%
      \colorrgb{0.00,0.00,0.00}%
      \put(682,1422){\makebox(0,0)[r]{\strut{}$5$}}%
      \colorrgb{0.00,0.00,0.00}%
      \put(682,1701){\makebox(0,0)[r]{\strut{}$10$}}%
      \colorrgb{0.00,0.00,0.00}%
      \put(682,1979){\makebox(0,0)[r]{\strut{}$15$}}%
      \colorrgb{0.00,0.00,0.00}%
      \put(682,2258){\makebox(0,0)[r]{\strut{}$20$}}%
      \colorrgb{0.00,0.00,0.00}%
      \put(682,2536){\makebox(0,0)[r]{\strut{}$25$}}%
      \colorrgb{0.00,0.00,0.00}%
      \put(682,2814){\makebox(0,0)[r]{\strut{}$30$}}%
      \colorrgb{0.00,0.00,0.00}%
      \put(682,3093){\makebox(0,0)[r]{\strut{}$35$}}%
      \colorrgb{0.00,0.00,0.00}%
      \put(682,3371){\makebox(0,0)[r]{\strut{}$40$}}%
      \colorrgb{0.00,0.00,0.00}%
      \put(814,924){\makebox(0,0){\strut{}$2006$}}%
      \colorrgb{0.00,0.00,0.00}%
      \put(2172,924){\makebox(0,0){\strut{}$2008$}}%
      \colorrgb{0.00,0.00,0.00}%
      \put(3530,924){\makebox(0,0){\strut{}$2010$}}%
      \colorrgb{0.00,0.00,0.00}%
      \put(4888,924){\makebox(0,0){\strut{}$2012$}}%
      \colorrgb{0.00,0.00,0.00}%
      \put(6246,924){\makebox(0,0){\strut{}$2014$}}%
      \colorrgb{0.00,0.00,0.00}%
      \put(7604,924){\makebox(0,0){\strut{}$2016$}}%
      \colorrgb{0.00,0.00,0.00}%
      \put(8962,924){\makebox(0,0){\strut{}$2018$}}%
    }%
    \gplgaddtomacro\gplfronttext{%
      \csname LTb\endcsname%
      \put(176,2257){\rotatebox{-270}{\makebox(0,0){\strut{}\% increase}}}%
      \put(4888,594){\makebox(0,0){\strut{}Year}}%
      \put(4888,3701){\makebox(0,0){\strut{}Electricity price increases in South Africa}}%
      \csname LTb\endcsname%
      \put(6770,393){\makebox(0,0)[r]{\strut{}Eskom general tariff increases (\%)}}%
      \csname LTb\endcsname%
      \put(1493,1562){\makebox(0,0){\strut{}6.0}}%
      \put(2172,2759){\makebox(0,0){\strut{}27.5}}%
      \put(2851,2970){\makebox(0,0){\strut{}31.3}}%
      \put(3530,2608){\makebox(0,0){\strut{}24.8}}%
      \put(4209,2664){\makebox(0,0){\strut{}25.8}}%
      \put(4888,2118){\makebox(0,0){\strut{}16.0}}%
      \put(5567,1673){\makebox(0,0){\strut{}8.0}}%
      \put(6246,1673){\makebox(0,0){\strut{}8.0}}%
      \put(6925,1673){\makebox(0,0){\strut{}8.0}}%
      \put(7604,1673){\makebox(0,0){\strut{}8.0}}%
      \put(8283,1673){\makebox(0,0){\strut{}8.0}}%
      \csname LTb\endcsname%
      \put(6770,173){\makebox(0,0)[r]{\strut{}Inflation rate in south Africa (\%)}}%
    }%
    \gplbacktext
    \put(0,0){\includegraphics{Graphs/1/Eskom/Eskom}}%
    \gplfronttext
  \end{picture}%
\endgroup
}
	\caption[Electricity price increases between 2007 and 2017.]{Electricity price increases between 2007 and 2017 \cite{Eskom2013Tariffs,Inflation2013}.}
	\label{fig: Eskom tariffs}
\end{figure}
\par
In addition to rising electricity costs, gold ore grades of South African mines have fallen substantially over the last few decades\cite{mudd2007global}. As ore grades decline, the energy utilised per unit of metal increases exponentially \cite{muller2010numerical}. Therefore mines require significantly more energy per unit of metal produced. This combination of tariff increases and increased energy usage per unit have lead to significant rises in mining operation costs.
\subsection{Mining Process}
\texttt{Brief description of the mining process}
\subsection{Mining Services}

\paragraph{Energy usage}\leavevmode\\
The mining industry uses extensive amounts of energy. In South Africa, the industry utilizes approximately 15\% of the national electricity supplier's yearly output, of which, gold and platinum mines use 80\%.\cite{Eskom2010Energy}
\par
Figure \ref{fig: Energy Split} shows the division of energy within the mining industry. From the chart it is reasoned that energy can be reduced most effectively through implementing energy interventions on mining material handling, processing and compressed air systems.
\begin{figure}[h]
	\centering
	\fbox{\includegraphics[ trim=0.7cm 7cm 0.7cm 0.7cm,width=\textwidth]{Graphs/1/Sankey/Sankey}}
	\caption[The energy split for the south african mining industry.]{The energy split for the south african mining industry \cite{Eskom2010Energy}.}
	\label{fig: Energy Split}
\end{figure}

\section{Compressed air systems in mining}
\subsection{Compressed air in operation}
	Largely due to their reliability, versatilely and ease of use, the South African mining industry has installed extensive compressed air networks. These systems can have compressors with capacities of up to 15 \gls{mw} \cite{Marais2012PhD}.However, the supply of compressed air is a highly energy demanding and costly process \cite{padachi2009energy}.  The energy used for compressed air production contributes to between 9\% and 20\% of the total mining energy consumption	\cite{Eskom2010Energy,du2011development}. 
	\par
	Large compressed air systems are likely inefficient. Internationally, the expected energy savings potential of a large compressed air network is 15\% \cite{neale2009compressed}. Marais \cite{marais2013simplification} showed that energy savings of up to 30\% and 40\% can be attained through various interventions. 
	\paragraph{Pneumatic rock drills}\leavevmode\\
	 Drilling is mainly performed in the production areas or stopes of a mine. Drill machines are used to drill holes into the rock face. Once the holes have been drilled, explosives are then installed to break up the rock \cite{van2008development}.
	 \par
	  Compressed air is used to power pneumatic rock drills within a mine. Pneumatic rock drills run at an efficiency of 2\%. This is low when compared to alternative rock drills such as electric, oil electro-hydraulic and hydro-powered drills that run at an efficiency of between 20-31\% \cite{fraser2008saving,vanTonder2010Masters}. 
	  \par
	  Reduced pressure to the drill will cause it to run even less efficiently. A study by  Bester \textit{et al.} showed that between 2002 and 2013 compressed air and energy consumption per tonne of ore produced had steadily increased as shown in figure \ref{fig: Compressed energy and air flow per ton}. The increase in consumption per \gls{t} was a result of a reduction in air pressure at the mining stopes. Measurements indicated that the pressure was as low as 300 \gls{kpa}. This reduced the efficiency of the rock drills. Before 2002 pressure was maintained above 500 \gls{kpa} at the stopes.  .\cite{bester2013effect} \par
	\begin{figure}[h]
	\centering
	\fbox{% GNUPLOT: LaTeX picture with Postscript
\begingroup
  \makeatletter
  \providecommand\color[2][]{%
    \GenericError{(gnuplot) \space\space\space\@spaces}{%
      Package color not loaded in conjunction with
      terminal option `colourtext'%
    }{See the gnuplot documentation for explanation.%
    }{Either use 'blacktext' in gnuplot or load the package
      color.sty in LaTeX.}%
    \renewcommand\color[2][]{}%
  }%
  \providecommand\includegraphics[2][]{%
    \GenericError{(gnuplot) \space\space\space\@spaces}{%
      Package graphicx or graphics not loaded%
    }{See the gnuplot documentation for explanation.%
    }{The gnuplot epslatex terminal needs graphicx.sty or graphics.sty.}%
    \renewcommand\includegraphics[2][]{}%
  }%
  \providecommand\rotatebox[2]{#2}%
  \@ifundefined{ifGPcolor}{%
    \newif\ifGPcolor
    \GPcolortrue
  }{}%
  \@ifundefined{ifGPblacktext}{%
    \newif\ifGPblacktext
    \GPblacktextfalse
  }{}%
  % define a \g@addto@macro without @ in the name:
  \let\gplgaddtomacro\g@addto@macro
  % define empty templates for all commands taking text:
  \gdef\gplbacktext{}%
  \gdef\gplfronttext{}%
  \makeatother
  \ifGPblacktext
    % no textcolor at all
    \def\colorrgb#1{}%
    \def\colorgray#1{}%
  \else
    % gray or color?
    \ifGPcolor
      \def\colorrgb#1{\color[rgb]{#1}}%
      \def\colorgray#1{\color[gray]{#1}}%
      \expandafter\def\csname LTw\endcsname{\color{white}}%
      \expandafter\def\csname LTb\endcsname{\color{black}}%
      \expandafter\def\csname LTa\endcsname{\color{black}}%
      \expandafter\def\csname LT0\endcsname{\color[rgb]{1,0,0}}%
      \expandafter\def\csname LT1\endcsname{\color[rgb]{0,1,0}}%
      \expandafter\def\csname LT2\endcsname{\color[rgb]{0,0,1}}%
      \expandafter\def\csname LT3\endcsname{\color[rgb]{1,0,1}}%
      \expandafter\def\csname LT4\endcsname{\color[rgb]{0,1,1}}%
      \expandafter\def\csname LT5\endcsname{\color[rgb]{1,1,0}}%
      \expandafter\def\csname LT6\endcsname{\color[rgb]{0,0,0}}%
      \expandafter\def\csname LT7\endcsname{\color[rgb]{1,0.3,0}}%
      \expandafter\def\csname LT8\endcsname{\color[rgb]{0.5,0.5,0.5}}%
    \else
      % gray
      \def\colorrgb#1{\color{black}}%
      \def\colorgray#1{\color[gray]{#1}}%
      \expandafter\def\csname LTw\endcsname{\color{white}}%
      \expandafter\def\csname LTb\endcsname{\color{black}}%
      \expandafter\def\csname LTa\endcsname{\color{black}}%
      \expandafter\def\csname LT0\endcsname{\color{black}}%
      \expandafter\def\csname LT1\endcsname{\color{black}}%
      \expandafter\def\csname LT2\endcsname{\color{black}}%
      \expandafter\def\csname LT3\endcsname{\color{black}}%
      \expandafter\def\csname LT4\endcsname{\color{black}}%
      \expandafter\def\csname LT5\endcsname{\color{black}}%
      \expandafter\def\csname LT6\endcsname{\color{black}}%
      \expandafter\def\csname LT7\endcsname{\color{black}}%
      \expandafter\def\csname LT8\endcsname{\color{black}}%
    \fi
  \fi
    \setlength{\unitlength}{0.0500bp}%
    \ifx\gptboxheight\undefined%
      \newlength{\gptboxheight}%
      \newlength{\gptboxwidth}%
      \newsavebox{\gptboxtext}%
    \fi%
    \setlength{\fboxrule}{0.5pt}%
    \setlength{\fboxsep}{1pt}%
\begin{picture}(9360.00,4032.00)%
    \gplgaddtomacro\gplbacktext{%
      \colorrgb{0.00,0.00,0.00}%
      \put(682,924){\makebox(0,0)[r]{\strut{}$0$}}%
      \colorrgb{0.00,0.00,0.00}%
      \put(682,1230){\makebox(0,0)[r]{\strut{}$5$}}%
      \colorrgb{0.00,0.00,0.00}%
      \put(682,1536){\makebox(0,0)[r]{\strut{}$10$}}%
      \colorrgb{0.00,0.00,0.00}%
      \put(682,1842){\makebox(0,0)[r]{\strut{}$15$}}%
      \colorrgb{0.00,0.00,0.00}%
      \put(682,2148){\makebox(0,0)[r]{\strut{}$20$}}%
      \colorrgb{0.00,0.00,0.00}%
      \put(682,2453){\makebox(0,0)[r]{\strut{}$25$}}%
      \colorrgb{0.00,0.00,0.00}%
      \put(682,2759){\makebox(0,0)[r]{\strut{}$30$}}%
      \colorrgb{0.00,0.00,0.00}%
      \put(682,3065){\makebox(0,0)[r]{\strut{}$35$}}%
      \colorrgb{0.00,0.00,0.00}%
      \put(682,3371){\makebox(0,0)[r]{\strut{}$40$}}%
      \colorrgb{0.00,0.00,0.00}%
      \put(814,704){\makebox(0,0){\strut{}$2002$}}%
      \colorrgb{0.00,0.00,0.00}%
      \put(2025,704){\makebox(0,0){\strut{}$2004$}}%
      \colorrgb{0.00,0.00,0.00}%
      \put(3237,704){\makebox(0,0){\strut{}$2006$}}%
      \colorrgb{0.00,0.00,0.00}%
      \put(4448,704){\makebox(0,0){\strut{}$2008$}}%
      \colorrgb{0.00,0.00,0.00}%
      \put(5659,704){\makebox(0,0){\strut{}$2010$}}%
      \colorrgb{0.00,0.00,0.00}%
      \put(6871,704){\makebox(0,0){\strut{}$2012$}}%
      \colorrgb{0.00,0.00,0.00}%
      \put(8082,704){\makebox(0,0){\strut{}$2014$}}%
      \colorrgb{0.00,0.00,0.00}%
      \put(8214,924){\makebox(0,0)[l]{\strut{}$0$}}%
      \colorrgb{0.00,0.00,0.00}%
      \put(8214,1230){\makebox(0,0)[l]{\strut{}$50$}}%
      \colorrgb{0.00,0.00,0.00}%
      \put(8214,1536){\makebox(0,0)[l]{\strut{}$100$}}%
      \colorrgb{0.00,0.00,0.00}%
      \put(8214,1842){\makebox(0,0)[l]{\strut{}$150$}}%
      \colorrgb{0.00,0.00,0.00}%
      \put(8214,2148){\makebox(0,0)[l]{\strut{}$200$}}%
      \colorrgb{0.00,0.00,0.00}%
      \put(8214,2453){\makebox(0,0)[l]{\strut{}$250$}}%
      \colorrgb{0.00,0.00,0.00}%
      \put(8214,2759){\makebox(0,0)[l]{\strut{}$300$}}%
      \colorrgb{0.00,0.00,0.00}%
      \put(8214,3065){\makebox(0,0)[l]{\strut{}$350$}}%
      \colorrgb{0.00,0.00,0.00}%
      \put(8214,3371){\makebox(0,0)[l]{\strut{}$400$}}%
    }%
    \gplgaddtomacro\gplfronttext{%
      \csname LTb\endcsname%
      \put(176,2147){\rotatebox{-270}{\makebox(0,0){\strut{}kWh/t}}}%
      \put(8851,2147){\rotatebox{-270}{\makebox(0,0){\strut{}$m^3$/t}}}%
      \put(4448,374){\makebox(0,0){\strut{}Year}}%
      \put(4448,3701){\makebox(0,0){\strut{}Compressed air energy and Volume consumed per ton}}%
      \csname LTb\endcsname%
      \put(3593,173){\makebox(0,0)[r]{\strut{}Energy per Ton (kWh/t)}}%
      \csname LTb\endcsname%
      \put(7352,173){\makebox(0,0)[r]{\strut{}Volume per Ton ($m^3$/t)}}%
    }%
    \gplbacktext
    \put(0,0){\includegraphics{Graphs/1/EVperT/EVperT}}%
    \gplfronttext
  \end{picture}%
\endgroup
}
	\caption[The Compressed air energy and flow consumed per T of ore produced.]{The Compressed air energy and flow consumed per T of ore produced. Adopted from Bester \textit{et al.} \cite{bester2013effect}.}
	\label{fig: Compressed energy and air flow per ton}
	\end{figure}
	\paragraph{Refuge bays}\leavevmode\\
	Refuge bays are installed in mines to provide safety to miners during emergencies. Most mines will utilise compressed air to deliver safe, cool air to the refuge bay. The provision of 1.42 $l/s$ of air per person at a pressure between 200 and 300 \gls{kpa} is required to provide oxygen and prevent any poisonousness gas entering the refuge \cite{brake1999criteria}.\par
	Airflow in the refuge bays can be controlled with a manual valve within the chamber. Any many cases, this valve is often misused by mine workers who open the valves fully in order to cool the bay through the decompression of the air. \texttt{*** need source ***}
	\paragraph{Compressed air control}\leavevmode\\
	On many large compressed air networks, the intake guide vain position on a compressor is manipulated in order to obtain the air flow and pressure requirements. Typically, guide vains are opened and closed to increase and decrease the compressors discharge pressure. When more pressure is required than can be obtained with the guide vains fully opened, another compressor is needed to operate.\par
	As shown in the figure \ref{fig: Guide vain position}, showing guide vain positions vs power for a typical mining compressor, the guide vains are controlled between 40\% and 100\% of their fully open position. Reducing and increasing the guide vain position will effect the power output of the compressor.The relationship between power and guide vain position of a compressor can be approximated as a linear function. A guide vain position of 40\% will relate to an output power of about 60\% of maximum power.
	\begin{figure}[h]
		\centering
		\fbox{\input{Graphs/1/GuideVainPosition/GuideVainPosition}}
		\caption[The relation between guide vain position and compressor output power.]{The relation between guide vain position and compressor output power.}
		\label{fig: Guide vain position}
	\end{figure}
	\paragraph{Operation schedule}\leavevmode\\
	On a typical mine, various operations will take place at different times of the day. Depending on the activity taking place, many mines will control the pressure to meet the requirements \cite{Kriel2014Masters,Marais2012PhD}. Figure \ref{fig: Mining schedule} shows the schedule and pressure requirement on a typical deep level mine.\par 
	As shown in the figure, the pressure requirement changes depending on the activity taking place. The drilling shift typically has the highest pressure requirement whilst blasting shift requires the lowest. Schedules and operation philosophies can differ between mines. Different operational schedules require alternative pressure requirement profiles.
	\begin{figure}[h]
		\centering
		\fbox{% GNUPLOT: LaTeX picture with Postscript
\begingroup
  \makeatletter
  \providecommand\color[2][]{%
    \GenericError{(gnuplot) \space\space\space\@spaces}{%
      Package color not loaded in conjunction with
      terminal option `colourtext'%
    }{See the gnuplot documentation for explanation.%
    }{Either use 'blacktext' in gnuplot or load the package
      color.sty in LaTeX.}%
    \renewcommand\color[2][]{}%
  }%
  \providecommand\includegraphics[2][]{%
    \GenericError{(gnuplot) \space\space\space\@spaces}{%
      Package graphicx or graphics not loaded%
    }{See the gnuplot documentation for explanation.%
    }{The gnuplot epslatex terminal needs graphicx.sty or graphics.sty.}%
    \renewcommand\includegraphics[2][]{}%
  }%
  \providecommand\rotatebox[2]{#2}%
  \@ifundefined{ifGPcolor}{%
    \newif\ifGPcolor
    \GPcolortrue
  }{}%
  \@ifundefined{ifGPblacktext}{%
    \newif\ifGPblacktext
    \GPblacktextfalse
  }{}%
  % define a \g@addto@macro without @ in the name:
  \let\gplgaddtomacro\g@addto@macro
  % define empty templates for all commands taking text:
  \gdef\gplbacktext{}%
  \gdef\gplfronttext{}%
  \makeatother
  \ifGPblacktext
    % no textcolor at all
    \def\colorrgb#1{}%
    \def\colorgray#1{}%
  \else
    % gray or color?
    \ifGPcolor
      \def\colorrgb#1{\color[rgb]{#1}}%
      \def\colorgray#1{\color[gray]{#1}}%
      \expandafter\def\csname LTw\endcsname{\color{white}}%
      \expandafter\def\csname LTb\endcsname{\color{black}}%
      \expandafter\def\csname LTa\endcsname{\color{black}}%
      \expandafter\def\csname LT0\endcsname{\color[rgb]{1,0,0}}%
      \expandafter\def\csname LT1\endcsname{\color[rgb]{0,1,0}}%
      \expandafter\def\csname LT2\endcsname{\color[rgb]{0,0,1}}%
      \expandafter\def\csname LT3\endcsname{\color[rgb]{1,0,1}}%
      \expandafter\def\csname LT4\endcsname{\color[rgb]{0,1,1}}%
      \expandafter\def\csname LT5\endcsname{\color[rgb]{1,1,0}}%
      \expandafter\def\csname LT6\endcsname{\color[rgb]{0,0,0}}%
      \expandafter\def\csname LT7\endcsname{\color[rgb]{1,0.3,0}}%
      \expandafter\def\csname LT8\endcsname{\color[rgb]{0.5,0.5,0.5}}%
    \else
      % gray
      \def\colorrgb#1{\color{black}}%
      \def\colorgray#1{\color[gray]{#1}}%
      \expandafter\def\csname LTw\endcsname{\color{white}}%
      \expandafter\def\csname LTb\endcsname{\color{black}}%
      \expandafter\def\csname LTa\endcsname{\color{black}}%
      \expandafter\def\csname LT0\endcsname{\color{black}}%
      \expandafter\def\csname LT1\endcsname{\color{black}}%
      \expandafter\def\csname LT2\endcsname{\color{black}}%
      \expandafter\def\csname LT3\endcsname{\color{black}}%
      \expandafter\def\csname LT4\endcsname{\color{black}}%
      \expandafter\def\csname LT5\endcsname{\color{black}}%
      \expandafter\def\csname LT6\endcsname{\color{black}}%
      \expandafter\def\csname LT7\endcsname{\color{black}}%
      \expandafter\def\csname LT8\endcsname{\color{black}}%
    \fi
  \fi
    \setlength{\unitlength}{0.0500bp}%
    \ifx\gptboxheight\undefined%
      \newlength{\gptboxheight}%
      \newlength{\gptboxwidth}%
      \newsavebox{\gptboxtext}%
    \fi%
    \setlength{\fboxrule}{0.5pt}%
    \setlength{\fboxsep}{1pt}%
\begin{picture}(9360.00,4032.00)%
    \gplgaddtomacro\gplbacktext{%
      \colorrgb{0.42,0.42,0.42}%
      \put(814,1364){\makebox(0,0)[r]{\strut{}$500$}}%
      \colorrgb{0.42,0.42,0.42}%
      \put(814,2033){\makebox(0,0)[r]{\strut{}$520$}}%
      \colorrgb{0.42,0.42,0.42}%
      \put(814,2702){\makebox(0,0)[r]{\strut{}$540$}}%
      \colorrgb{0.42,0.42,0.42}%
      \put(814,3371){\makebox(0,0)[r]{\strut{}$560$}}%
      \colorrgb{0.42,0.42,0.42}%
      \put(946,1232){\rotatebox{-270}{\makebox(0,0)[r]{\strut{}00:00}}}%
      \colorrgb{0.42,0.42,0.42}%
      \put(1295,1232){\rotatebox{-270}{\makebox(0,0)[r]{\strut{}01:00}}}%
      \colorrgb{0.42,0.42,0.42}%
      \put(1643,1232){\rotatebox{-270}{\makebox(0,0)[r]{\strut{}02:00}}}%
      \colorrgb{0.42,0.42,0.42}%
      \put(1992,1232){\rotatebox{-270}{\makebox(0,0)[r]{\strut{}03:00}}}%
      \colorrgb{0.42,0.42,0.42}%
      \put(2340,1232){\rotatebox{-270}{\makebox(0,0)[r]{\strut{}04:00}}}%
      \colorrgb{0.42,0.42,0.42}%
      \put(2689,1232){\rotatebox{-270}{\makebox(0,0)[r]{\strut{}05:00}}}%
      \colorrgb{0.42,0.42,0.42}%
      \put(3037,1232){\rotatebox{-270}{\makebox(0,0)[r]{\strut{}06:00}}}%
      \colorrgb{0.42,0.42,0.42}%
      \put(3386,1232){\rotatebox{-270}{\makebox(0,0)[r]{\strut{}07:00}}}%
      \colorrgb{0.42,0.42,0.42}%
      \put(3734,1232){\rotatebox{-270}{\makebox(0,0)[r]{\strut{}08:00}}}%
      \colorrgb{0.42,0.42,0.42}%
      \put(4083,1232){\rotatebox{-270}{\makebox(0,0)[r]{\strut{}09:00}}}%
      \colorrgb{0.42,0.42,0.42}%
      \put(4431,1232){\rotatebox{-270}{\makebox(0,0)[r]{\strut{}10:00}}}%
      \colorrgb{0.42,0.42,0.42}%
      \put(4780,1232){\rotatebox{-270}{\makebox(0,0)[r]{\strut{}11:00}}}%
      \colorrgb{0.42,0.42,0.42}%
      \put(5128,1232){\rotatebox{-270}{\makebox(0,0)[r]{\strut{}12:00}}}%
      \colorrgb{0.42,0.42,0.42}%
      \put(5477,1232){\rotatebox{-270}{\makebox(0,0)[r]{\strut{}13:00}}}%
      \colorrgb{0.42,0.42,0.42}%
      \put(5825,1232){\rotatebox{-270}{\makebox(0,0)[r]{\strut{}14:00}}}%
      \colorrgb{0.42,0.42,0.42}%
      \put(6174,1232){\rotatebox{-270}{\makebox(0,0)[r]{\strut{}15:00}}}%
      \colorrgb{0.42,0.42,0.42}%
      \put(6522,1232){\rotatebox{-270}{\makebox(0,0)[r]{\strut{}16:00}}}%
      \colorrgb{0.42,0.42,0.42}%
      \put(6871,1232){\rotatebox{-270}{\makebox(0,0)[r]{\strut{}17:00}}}%
      \colorrgb{0.42,0.42,0.42}%
      \put(7219,1232){\rotatebox{-270}{\makebox(0,0)[r]{\strut{}18:00}}}%
      \colorrgb{0.42,0.42,0.42}%
      \put(7568,1232){\rotatebox{-270}{\makebox(0,0)[r]{\strut{}19:00}}}%
      \colorrgb{0.42,0.42,0.42}%
      \put(7916,1232){\rotatebox{-270}{\makebox(0,0)[r]{\strut{}20:00}}}%
      \colorrgb{0.42,0.42,0.42}%
      \put(8265,1232){\rotatebox{-270}{\makebox(0,0)[r]{\strut{}21:00}}}%
      \colorrgb{0.42,0.42,0.42}%
      \put(8613,1232){\rotatebox{-270}{\makebox(0,0)[r]{\strut{}22:00}}}%
      \colorrgb{0.42,0.42,0.42}%
      \put(8962,1232){\rotatebox{-270}{\makebox(0,0)[r]{\strut{}23:00}}}%
    }%
    \gplgaddtomacro\gplfronttext{%
      \csname LTb\endcsname%
      \put(176,2367){\rotatebox{-270}{\makebox(0,0){\strut{}kPa}}}%
      \put(4954,374){\makebox(0,0){\strut{}Time of day}}%
      \put(4954,3701){\makebox(0,0){\strut{}Typical mining schedule and pressure requirement}}%
      \csname LTb\endcsname%
      \put(6242,173){\makebox(0,0)[r]{\strut{}Pressure requirement (kPa)}}%
      \csname LTb\endcsname%
      \put(1817,2368){\rotatebox{-270}{\makebox(0,0){\strut{}\shortstack{Sweeping and \\ cleaning}}}}%
      \put(3037,2368){\rotatebox{-270}{\makebox(0,0){\strut{}\shortstack{Workers travel to \\ working areas}}}}%
      \put(4605,2368){\makebox(0,0){\strut{}\shortstack{Drilling}}}%
      \put(6174,2368){\rotatebox{-270}{\makebox(0,0){\strut{}\shortstack{Explosive charge \\ up}}}}%
      \put(7219,2368){\makebox(0,0){\strut{}\shortstack{Blasting}}}%
      \put(8613,2368){\rotatebox{-270}{\makebox(0,0){\strut{}\shortstack{Sweeping and \\ cleaning}}}}%
    }%
    \gplbacktext
    \put(0,0){\includegraphics{MiningSchedule}}%
    \gplfronttext
  \end{picture}%
\endgroup
}
		\caption[A typical operation schedule of a deep level mine.]{The typical operation schedule of a deep level mine \cite{Kriel2014Masters}.}
		\label{fig: Mining schedule}
	\end{figure}
	\subsection{Characteristic inefficiencies}
	Compressed air distribution networks in the mining industry consist of multiple compressors and working areas up to eight kilometres away from the source \cite{Marais2012PhD}. Due to their size and complexity, these systems are prone to large energy losses.
	\par 
	Compressed air leakage accounts for as much as 35\% of the energy losses of a compressed air network \cite{Lawrence2004Improving}. Other systemic losses include, faulty valves, pipe diameter fluctuations, obstructed air compressor intake filters and inefficient compressors. 	
	\subsection{Inefficiency identification methods}
	Leakage and inefficiency detection strategies is not often pursued in the South African mining industry \cite{vanTonder2010Masters}. Many mines do however perform leak inspections either internally or by a outside company. In these inspections, an ultrasonic detector is used to locate the leak. Alternatively, some mines employ the \enquote{walk and listen} method to identify leaks from the audible sound that it produces \cite{vanTonder2010Masters}. Once the inspection is completed, the findings, including the locations and estimated costs of all identified leaks, are reported.
	\subsection{Instrumentation and measurements}
	For large industrial systems, thorough instrumentation is necessary in order to monitor performance and condition throughout the system. In a mining compressed air network, instrumentation is installed on the compressor to monitor flows, pressures, temperatures and guide vain position. Electrical instrumentation is also installed for sensing currents, power factors, voltages and power. On control valves, input/output pressures, flows and valve position are usually measured with instrumentation.	
	\par
	\Glspl{plc}. A \gls{scada} system is is used to observe and monitor all of the instrumentation. 
	\par
	When instrumentation is not installed. Manual measurements are performed...
	\texttt{Need to added a citation or two}
	\subsection{Compressed air savings strategies}
	The main strategies fo reducing energy on compressed air systems are summarised as follows \cite{Snyman2011Masters}:
	\begin{itemize}
		\item Reducing leaks.
		\item Reducing demand.
		\item Reducing unauthorised usage.
		\item Increasing supply efficiency.
		\item Optimising supply.
	\end{itemize}
\section{Simulations in industry}
Continuous improvements in computing hardware has led to major advancement in software technology. Consequently the use of computational simulation has become an increasingly valuable tool for many industries.\cite{kocsis2003integration} \par 
 In \textit{ Handbook of simulation: principles, methodology, advances, applications, and practice}, the advantages of the use of simulation in industry are discussed as follows \cite{banks1998handbook}: 
\begin{itemize}
	\item The ability to test new policies, operating procedures and methods without causing a disruption to the actual system.
	\item The means to identify problems in complex systems by gathering insight in the interactions within the system.
	\item The facility to compress or expand time to investigate phenomena thoroughly.
	\item The capability to determine the limits and constraints within a system.
	\item The potential to build consensus with regard to proposed designs or modifications.
\end{itemize}
\subsection{Thermal-hydraulic simulation}
\gls{ths} is the modelling and computational analysis of Thermal-hydraulic systems. \gls{ths} models can be developed large mining systems such as cooling ,compressed air water reticulation etc..\par

\subsection{Use in mining}
In industry, various software packages are used for modelling and simulating these systems. Two such packages are KY-pipe and Simulation Toolbox.

\section{Problem statement and objectives}
\texttt{Identification of research problem and formulation of objectives \\should be unambiguous and intelligible}
\subsection{Problem statement}
\subsection{Research objectives}
\section{Dissertation overview}
\texttt{Describe (in approximately one sentence each) the contents of \\each of the dissertation chapters. No results here.}